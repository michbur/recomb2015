\documentclass[final]{beamer}\usepackage[]{graphicx}\usepackage[]{color}
%% maxwidth is the original width if it is less than linewidth
%% otherwise use linewidth (to make sure the graphics do not exceed the margin)
\makeatletter
\def\maxwidth{ %
  \ifdim\Gin@nat@width>\linewidth
    \linewidth
  \else
    \Gin@nat@width
  \fi
}
\makeatother

\definecolor{fgcolor}{rgb}{0.345, 0.345, 0.345}
\newcommand{\hlnum}[1]{\textcolor[rgb]{0.686,0.059,0.569}{#1}}%
\newcommand{\hlstr}[1]{\textcolor[rgb]{0.192,0.494,0.8}{#1}}%
\newcommand{\hlcom}[1]{\textcolor[rgb]{0.678,0.584,0.686}{\textit{#1}}}%
\newcommand{\hlopt}[1]{\textcolor[rgb]{0,0,0}{#1}}%
\newcommand{\hlstd}[1]{\textcolor[rgb]{0.345,0.345,0.345}{#1}}%
\newcommand{\hlkwa}[1]{\textcolor[rgb]{0.161,0.373,0.58}{\textbf{#1}}}%
\newcommand{\hlkwb}[1]{\textcolor[rgb]{0.69,0.353,0.396}{#1}}%
\newcommand{\hlkwc}[1]{\textcolor[rgb]{0.333,0.667,0.333}{#1}}%
\newcommand{\hlkwd}[1]{\textcolor[rgb]{0.737,0.353,0.396}{\textbf{#1}}}%

\usepackage{framed}
\makeatletter
\newenvironment{kframe}{%
 \def\at@end@of@kframe{}%
 \ifinner\ifhmode%
  \def\at@end@of@kframe{\end{minipage}}%
  \begin{minipage}{\columnwidth}%
 \fi\fi%
 \def\FrameCommand##1{\hskip\@totalleftmargin \hskip-\fboxsep
 \colorbox{shadecolor}{##1}\hskip-\fboxsep
     % There is no \\@totalrightmargin, so:
     \hskip-\linewidth \hskip-\@totalleftmargin \hskip\columnwidth}%
 \MakeFramed {\advance\hsize-\width
   \@totalleftmargin\z@ \linewidth\hsize
   \@setminipage}}%
 {\par\unskip\endMakeFramed%
 \at@end@of@kframe}
\makeatother

\definecolor{shadecolor}{rgb}{.97, .97, .97}
\definecolor{messagecolor}{rgb}{0, 0, 0}
\definecolor{warningcolor}{rgb}{1, 0, 1}
\definecolor{errorcolor}{rgb}{1, 0, 0}
\newenvironment{knitrout}{}{} % an empty environment to be redefined in TeX

\usepackage{alltt}
\usepackage{grffile}
\mode<presentation>{\usetheme{CambridgeUSPOL}}

\usepackage[utf8]{inputenc}
\usepackage{amsfonts}
\usepackage{amsmath}
\usepackage{natbib}
\usepackage{graphicx}
\usepackage{array,booktabs,tabularx}
\newcolumntype{Z}{>{\centering\arraybackslash}X}

% rysunki
\usepackage{tikz}
\usepackage{ifthen}
\usepackage{xxcolor}
\usetikzlibrary{arrows}
\usetikzlibrary[topaths]
\usetikzlibrary{decorations.pathreplacing}
%\usepackage{times}\usefonttheme{professionalfonts}  % times is obsolete
\usefonttheme[onlymath]{serif}
\boldmath
\usepackage[orientation=portrait,size=a0,scale=1.4,debug]{beamerposter}                       % e.g. for DIN-A0 poster
%\usepackage[orientation=portrait,size=a1,scale=1.4,grid,debug]{beamerposter}                  % e.g. for DIN-A1 poster, with optional grid and debug output
%\usepackage[size=custom,width=200,height=120,scale=2,debug]{beamerposter}                     % e.g. for custom size poster
%\usepackage[orientation=portrait,size=a0,scale=1.0,printer=rwth-glossy-uv.df]{beamerposter}   % e.g. for DIN-A0 poster with rwth-glossy-uv printer check
% ...
%

\usecolortheme{seagull}
\useinnertheme{rectangles}
\setbeamercolor{item projected}{bg=darkred}
% \setbeamertemplate{enumerate items}[default]
\setbeamertemplate{caption}{\insertcaption} 
\setbeamertemplate{navigation symbols}{}
\setbeamercovered{transparent}
\setbeamercolor{block title}{fg=darkred}
\setbeamercolor{local structure}{fg=darkred}

\setbeamercolor*{enumerate item}{fg=darkred}
\setbeamercolor*{enumerate subitem}{fg=darkred}
\setbeamercolor*{enumerate subsubitem}{fg=darkred}

\setbeamercolor*{itemize item}{fg=darkred}
\setbeamercolor*{itemize subitem}{fg=darkred}
\setbeamercolor*{itemize subsubitem}{fg=darkred}

\newlength{\columnheight}
\setlength{\columnheight}{93cm}
\renewcommand{\thetable}{}
\def\andname{,}
\authornote{}

\renewcommand{\APACrefatitle}[2]{}
\renewcommand{\bibliographytypesize}{\footnotesize} 
\renewcommand{\APACrefYearMonthDay}[3]{%
  {\BBOP}{#1}
  {\BBCP}
}
\IfFileExists{upquote.sty}{\usepackage{upquote}}{}
\begin{document}






\date{}
\author{Piotr Sobczyk\inst{1}*, Micha\l{}  Burdukiewicz\inst{2}, Chris Lauber\inst{3}, Pawe\l{} Mackiewicz\inst{2} \\
*Piotr.Sobczyk@pwr.edu.pl}


\institute{\small{

\textsuperscript{1}Wroc\l{}aw University of Technology, Department of Mathematics, Poland

\vspace{0.3cm}

\textsuperscript{2}University of Wroc\l{}aw, Department of Genomics, Poland 

\vspace{0.3cm}

\textsuperscript{3}Dresden University of Technology, Institute of Medical Informatics and Biometry, Poland 

}}
\title{\huge Quick Permutation Test: feature filtering of n-gram data}

\begin{frame}
  \begin{columns}
    \begin{column}{.49\textwidth}
      \begin{beamercolorbox}[center,wd=\textwidth]{postercolumn}
        \begin{minipage}[T]{.95\textwidth}
          \parbox[t][\columnheight]{\textwidth}
            {
    
        
    \begin{block}{Introduction}
      N-grams (k-tuples) are vectors of n characters derived from input sequence(s). They may form continuous sub-sequences or be discontinuous. Another important n-gram parameter is its position. Instead of just counting n-grams, one may want to count how many n-grams occur at a given position in multiple (e.g. related) sequences.

Originally developed for natural language processing, n-grams are also used in genomics~\citep{fang2011}, transcriptomics~\citep{wang2014} and proteomics~\citep{guo2014}.

\small{
       \begin{columns}[c] % the "c" option specifies center vertical alignment
    \column{.5\textwidth} 
% latex table generated in R 3.1.3 by xtable 1.7-4 package
% Thu Apr  9 10:13:50 2015
\begin{table}[ht]
\centering
\begin{tabular}{rrrrrrr}
  \hline
 & P1 & P2 & P3 & P4 & P5 & P6 \\ 
  \hline
S1 & 3 & 1 & 2 & 3 & 2 & 2 \\ 
  S2 & 3 & 3 & 2 & 2 & 4 & 2 \\ 
  S3 & 2 & 4 & 4 & 4 & 4 & 1 \\ 
   \hline
\end{tabular}
\caption{Sample sequences.} 
\end{table}

      
      
     % column designated by a command

    \column{.5\textwidth}
    
% latex table generated in R 3.1.3 by xtable 1.7-4 package
% Thu Apr  9 10:13:50 2015
\begin{table}[ht]
\centering
\begin{tabular}{rrrr}
  \hline
1 & 2 & 3 & 4 \\ 
  \hline
1 & 3 & 2 & 0 \\ 
  0 & 3 & 2 & 1 \\ 
  1 & 1 & 0 & 4 \\ 
   \hline
\end{tabular}
\caption{Unigram counts.} 
\end{table}


    \end{columns}

% latex table generated in R 3.1.3 by xtable 1.7-4 package
% Thu Apr  9 10:13:50 2015
\begin{table}[ht]
\centering
\begin{tabular}{rrrrrrrrrrrrr}
  \hline
P1\_1 & P2\_1 & P3\_1 & P4\_1 & P5\_1 & P6\_1 & P1\_2 & P2\_2 & P3\_2 & P4\_2 & P5\_2 & P6\_2 & P1\_3 \\ 
  \hline
0 & 1 & 0 & 0 & 0 & 0 & 0 & 0 & 1 & 0 & 1 & 1 & 1 \\ 
  0 & 0 & 0 & 0 & 0 & 0 & 0 & 0 & 1 & 1 & 0 & 1 & 1 \\ 
  0 & 0 & 0 & 0 & 0 & 1 & 1 & 0 & 0 & 0 & 0 & 0 & 0 \\ 
   \hline
\end{tabular}
\caption{A fraction of possible unigrams with position information.} 
\end{table}

}    
    \end{block}
    
    \vfill
    
    \begin{block}{Curse of dimensionality}
    
Even when we limit ourselves to only continuous positioned n-grams, feature space growths rapidly with the number of elements in n-gram ($n$) and length of the sequence ($L$).    
    
Number of possible positioned n-grams: 

\begin{center}
\scalebox{0.85}{
$
n_{\text{max}} = L \times m^n
$
}

\\


\scalebox{0.85}{  
\begin{knitrout}
\definecolor{shadecolor}{rgb}{0.969, 0.969, 0.969}\color{fgcolor}

{\centering \includegraphics[width=\maxwidth]{figure/unnamed-chunk-4-1} 

}



\end{knitrout}
}
\end{center}
    \end{block}
    \vfill
    
    
    \begin{block}{Feature selecting permutation tests}
    Model and statistic independent permutation tests can be used to filter features obtained through counting n-grams.
    
    During a permutation test class labels are randomly exchanged during computation of significance statistic. p-values are defined as:
    
\begin{center}
\scalebox{0.85}{
$      
\textnormal{p-value} = \frac{N_{T_P > T_R}}{N}
$
}
\end{center}

where $N_{T_P > T_R}$ is number of times when $T_P$ (permuted test statistic) was more extreme than $T_R$ (test statistic for non-permuted data).

Permutation tests are computationally expensive (especially precise estimation of low p-values, because the number of permutations is inversely proportional to the interval between p-values).
      
    \end{block}
    \vfill
    
    
\begin{block}{QuiPT idea}

If probability that target equals 1 is $p$ and probability that feature equals
1 is $q$ then we can compute the probability of given observations, eg.

\begin{center}
\scalebox{0.85}{
$P(Target, Feature) = (1,1)) = p \cdot q$
}
\end{center}

\\
% 
% \begin{center}
% \scalebox{0.85}{
% $P(Target, Feature) = (1,0)) = p \cdot (1-q)$
% }
% \end{center}
% 
% \\
% 
% \begin{center}
% \scalebox{0.85}{
% $P(Target, Feature) = (0,1)) = (1-p) \cdot q$
% }
% \end{center}
% 
% \\
% 
% \begin{center}
% \scalebox{0.85}{
% $P(Target, Feature) = (0,0)) = (1-p) \cdot (1-q)$
% }
% \end{center}
% 
% \\

Therefore another view at permutation test is that we get a contingency table,
which needs to be tested for independance.
  
\end{block}
\vfill

\begin{block}{Independence test}
    
\begin{split*}
  F(n_{1,1}, n_{1,0}, n_{0,1}, n_{0,0}) = 
  {n \choose n_{1,1}} (p\cdot q)^{n_{1,1}} 
  { n - n_{1,1} \choose n_{1,0}} (p\cdot (1-q))^{n_{1,0}} \\
  {n - n_{1,1} - n_{1,0} \choose n_{0,1}} ((1-p)\cdot q)^{n_{0,1}} 
  {n - n_{1,1} - n_{1,0} -n_{0,1} \choose n_{0,0}} ((1-p)\cdot (1-q))^{n_{0,0}}$$
\end{split}


This distribution comes with two constraints:
$n_{1,\cdot} = n_{1,1} + n_{1,0}$ and
$n_{\cdot, 1} = n_{1,1} + n_{0,1}$.
Thus, conditioning on $n_{1,1}$, we get hypergeometric distribution.

This is in fact exact two-sided Fisher's test. Information Gain is used here 
as a way of deciding which contingency tables are more extreme.

\end{block}
\vfill 
    
    }
        \end{minipage}
      \end{beamercolorbox}
    \end{column}
    
    
%new column ------------------------------------------------------    
    
    \begin{column}{.52\textwidth}
      \begin{beamercolorbox}[center,wd=\textwidth]{postercolumn}
        \begin{minipage}[T]{.95\textwidth}  
          \parbox[t][\columnheight]{\textwidth}
            {
     

 
\begin{block}{Computational cost}

The cost of performing QuiPT is equal to computing Information Gain and 
probability of occurence for $n_{1,1} + n_{0,1}$ contingency tables.

Suppose we consider 6-grams build on sequences of length 25 build of four 
characters. Then there are around 100,000 n-grams, features to test. 
This means that for Benjamini-Hochberg procedure, we need to calculate 
p-values with accuracy of $0.05 \times 10^{-5}$. This requires at least 
2 million permutations. Each permutation, apart from reshuffling labels, 
requires computation of IG. Since n-gram features are very sparse vectors,
QuiPT needs to evaluate only few contingency tables.

\end{block}
\vfill 
    
    \begin{block}{Contingency table representation}
    
    
    \begin{columns}
    \centering
    \column{.5\textwidth} 
    \scalebox{0.95}{
\begin{knitrout}
\definecolor{shadecolor}{rgb}{0.969, 0.969, 0.969}\color{fgcolor}

{\centering \includegraphics[width=\maxwidth]{figure/unnamed-chunk-5-1} 

}



\end{knitrout}
}
\column{.5\textwidth}
% latex table generated in R 3.1.3 by xtable 1.7-4 package
% Thu Apr  9 10:13:51 2015
\begin{table}[ht]
\centering
\begin{tabular}{|c|ccc|}
  \hline
 & Target & Feature & Freq \\ 
  \hline
1 & 0 & 0 &  40 \\ 
  2 & 1 & 0 &  10 \\ 
  3 & 0 & 1 &  25 \\ 
  4 & 1 & 1 &  40 \\ 
   \hline
\end{tabular}
\end{table}

    \end{columns}
    
    \end{block}
    \vfill
    
%     \begin{block}{Test power}
%     \centering
% \scalebox{0.95}{  
% <<echo = FALSE,message=FALSE,eval=FALSE,fig.align='center'>>=
% 
% three_n_power <- three_n_power[sapply(three_n_power, function(i) i[[6]][[3]][2]) > 150]
% 
% power <- Reduce("+", lapply(three_n_power, function(single_rep) {
%   #single replication dat
%   sr_dat <- do.call(rbind, lapply(single_rep, unlist))
%   sr_dat[, c(1, 3, 5)]/matrix(rep(c(4, 32, 192), 6), nrow = 6, byrow = TRUE)
% }))/length(three_n_power)
% 
% ggplot(tfres(power), aes(x = ngram, y = value, fill = sig)) + 
%   geom_bar(stat = "identity", position = "dodge") +
%   scale_y_continuous("Important features found (%)") +
%   scale_x_discrete("n-gram") +
%   scale_fill_discrete("Signal strength") +
%   poster_theme
% 
% 
% @
% }
% \end{block}
%     \vfill

\begin{block}{Simulation scheme}
\begin{enumerate}[1.]
\item Random 4000 sequences (20 nucleotides each). The half of the sequences has label 0.
\item Choose a single position between 3 and 18 (to avoid border cases).
\item Resample nucleotides at chosen position. The dominant nucletoide has probabilitiy of occurence $p_d = 0.25$. Other nucleotides have probability of occurence $p_o = (1 - p_d)/3  $. 
\item Perform QuiPT (Information Gain) and choose significant features (with p-value $< 0.001$).
\item Iterate steps 1-4 over other values of $p_d$ - 0.38, 0.51, 0.65, 0.78, 0.91.
\item Repeat steps 1-5 200 times.
\end{enumerate}
    \end{block}
    \vfill

\begin{block}{Power and False discoveries}
    \centering
\scalebox{1}{  
\begin{knitrout}
\definecolor{shadecolor}{rgb}{0.969, 0.969, 0.969}\color{fgcolor}

{\centering \includegraphics[width=\maxwidth]{figure/unnamed-chunk-7-1} 

}



\end{knitrout}
}

    
    \end{block}
    \vfill
      
     
    \begin{block}{Summary}
      Quick permutation test is a powerful and quick equivalent of permutation test in binary feature-binary target testing scenario.
    \end{block}
    \vfill 
    
        \begin{block}{Avaibility}
        \footnotesize{
      
      biogram R package: 
      
      \url{http://cran.r-project.org/web/packages/biogram/}
        }
    \end{block}
    \vfill 
     
     
    \begin{block}{Bibliography}
    \tiny{
      \bibliographystyle{apalike}
      \bibliography{biogram_poster}
    }
    \end{block}
    \vfill
            }
        \end{minipage}
      \end{beamercolorbox}
    \end{column}
  \end{columns}  
\end{frame}
\end{document}
